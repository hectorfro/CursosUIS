\documentclass[spanish,12pt]{article}
\usepackage[ansinew]{inputenc} % Acepta caracteres en castellano
\usepackage[spanish]{babel}    % silabea palabras castellanas
\usepackage{amsmath}
\usepackage{amsfonts}
\usepackage{amssymb}
\usepackage[colorlinks=true,urlcolor=blue,linkcolor=blue]{hyperref} % navega por el doc
\usepackage{graphicx}
\usepackage{epstopdf}
\usepackage{fancyhdr}  
\pagestyle{empty}

\setlength{\topmargin}{-1.25in}
\setlength{\textheight}{10in}
\setlength{\textwidth}{6.8in}
\setlength{\oddsidemargin}{-0.5in}


\pagestyle{empty}
\begin{document}
\parbox{3.1cm}{\vspace{1.0cm} \input figuras/uislogo.tex} 
\parbox{17cm}{\bf \center {\footnotesize FACULTAD DE CIENCIAS -- ESCUELA DE F�SICA  \\ \vspace{0.2cm}
F�SICA  II - Quiz 1 \\ \vspace{0.1cm} 21 de febrero de 2024} }
\begin{center}
\line(1,0){500} 
\end{center}
\vspace{0.1cm}


\begin{enumerate}

\item Una carga puntual $q_1 = +6.0$ nC est� situada en el punto $\left\langle -4.0; 0.0; 0.0 \right\rangle$ cm. Una segunda carga  puntual $q_2 = + 5.0$ nC est� situada en $\left\langle 5.0; 8.0; 0.0 \right\rangle$ cm 
\begin{enumerate}
\item �Cu�l es el campo el�ctrico total en el punto $A=\left\langle -4.0; 8.0; 0.0 \right\rangle$ cm debido a $q_1$ y $q_2$?
\item  Si una carga  $q_3 = -2.0$ nC se coloca en el lugar $A$ �Cu�l ser�a la fuerza sobre esta carga? 
\end{enumerate}

\end{enumerate}


\vspace{0.4cm}
\begin{center}
\line(1,0){500}
\end{center}
\hspace{0.6cm} H.H.

\end{document}
